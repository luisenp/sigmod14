\documentclass{article}

\usepackage{natbib}
\usepackage{hyperref}

\title{Project Proposal}

\author{
James Atwood and Luis Pineda \\ % alphabetical order
}

\begin{document}
\maketitle

We have been working on the SIGMOD 2014 programming challenge.  Briefly,
in this challenge we are provided with a large (relational) social
network data set and asked to implement four queries related to the
graph structure of this data.  The motivation for choosing this
project is twofold; first, there is a large (and quickly increasing)
volume of graph-structured data available today, and second,
developing efficient mechanisms for representing and querying graph
data is a challenging research problem that is currently the subject
of considerable interest in the database community.

Traditionally, research in databases has focused on the relational
model first proposed by Codd \cite{codd1970relational}.  This model
becomes awkward and inefficient when applied to graph data
\cite{rodriguez2011graph}, particularly for queries related to
complex structure (i.e., requiring more than nearest neighbors).  For
an example, please see \cite{he2008graphs} Figures 1 and 2.  More
recent work has proposed other data models and query languages that
more appropriately capture the rich structure evident in graph data;
for instance,
\cite{he2008graphs,sun2012efficient,low2010graphlab} (see
\cite{angles2008survey} for a survey of recent graph database
models).  

%However, much of this work is task oriented; for example, a system
%may optimize for path-related queries at the expense of subgraph
%isomorphism.  For the task at hand, it is unclear which existing
%technologies, if any, provide the best performance for the queries of
%this challenge.

Originally, we designed our implementation around the open-source
Neo4j\footnote{\url{http://www.neo4j.org/}} disk-based graph database
system.  We thought this system was appropriate because the queries in
the challenge are largely path-oriented, and it is unlikely that all
of the relevant data will fit in memory.  Neo4j makes use of the
\emph{ADI} index structure \cite[Chapter~6]{IanRobinson:2013ul}
described in \cite{wang2004scalable}, which is designed to facilitate
efficient edge support checking (that is, quickly finding edges) and
adjacent edge checking (that is, quickly finding edges that share a
node).  This index structure seemed well-suited to the task at hand
because it allows a graph on disk to be efficiently queried with
regards to path.

Our implementation of this approach performed very poorly, however.
An implicit assumption of this design was that the high fixed cost of
populating the graph database would be amortized over the large number
of queries against the data.  Instead, we found that the runtime was
dominated by reading and indexing the data rather than performing the
queries.  It seems likely that this undesirable behavior will be found
with other database systems; if fixed setup costs are amortized over
the lifetime of a database system measured in years, the cost of
establishing the database is trivial, so reducing this cost is likely
not a design goal.


%%% LUIS - please edit this %%%
We turned our attention to a simpler key-value approach.
Specifically, we indexed nodes and edges via a simple in-memory hash
table.  This approach both reduced the time it took to index the data
and improved the speed of the queries.

%Accordingly, we propose the following approach.  Our primary data
%abstraction will be an adjacency matrix over the nodes which
%constitute a network.  The nodes will themselves be an interface for
%the data provided by the challenge.  We will be investigating the
%particulars of the implementation of the adjacency matrix and node
%abstractions throughout the project. To be more concrete, a node could
%be an object representing a person, with fields for attributes like
%gender and age.  Or, a node could simply be a pointer to a query which
%retrieves the node's data from disk.  The adjacency matrix could be a
%simple $n$ by $n$ array, where $n$ is the total number of people in
%the dataset.  This adjacency matrix representation scales poorly; we
%will probably need to employ some sparse representation or other
%compression mechanism to maintain this structure in memory, or
%implement some mechanism for managing adjacency matrices that are only
%partially stored in memory, such as the \emph{ADI} structure described
%in \cite{wang2004scalable}. Finally, after developing appropriate
%implementations of the node and matrix abstractions, we will focus on
%developing efficient indexing and graph traversing mechanisms for
%servicing the four types of query that are the subject of the SIGMOD
%challenge.

%Neo4j is a Java system, so our implementation language will be Java.
%We will use the embedded interface of
%Neo4j\footnote{\url{http://docs.neo4j.org/chunked/stable/tutorials-java-embedded.html}}
%in order to construct an indexed graph database from the provided
%data.  This database will be constructed and queried using the Core
%API, which provides the lowest user-level abstractions in the Neo4j
%system, and is generally faster than the higher-level Traversal and
%Cypher APIs \cite[Chapter~6]{IanRobinson:2013ul}.

%The following pseudocode demonstrates how we will use the disk-based
%system to perform the first query, which finds the shortest path
%between two people p1 and p2 who have replied to each other's comments
%at least x times.


%More specifically, our implementation language will be Java.  The
%adjacency matrix representation will initially be implemented as a
%sparse Colt
%matrix \footnote{\url{http://acs.lbl.gov/software/colt/api/cern/colt/matrix/package-summary.html}}.

Our current pipeline is as follows.  When the program starts, a one
hash table for each edge and node type is populated.  These nodes
include people, comments, tags, organizations, and others, found in
files such as person.csv, comments.csv, and so forth. We then create
necessary edges by iterating over x\_has\_y.csv files; for instance,
to associate comments with their creators, we iterate over
comment\_hasCreator\_person.csv and add edges from each person to each
comment.  Note that only nodes and edges which are required in some
way by the queries will be added to the database.

%%% LUIS - is this still relevant? %%%

Additionally, each node type is indexed by properties that are 
relevant to the queries. Some examples are:


\begin{itemize}
\item Indexing tag nodes by id will be useful for performing efficient 
lookups for query 2.
\item Indexing organization and place nodes by id will be useful for 
performing efficient lookups for query 3.
\item Storing the number of common tags between two persons as part of 
the KNOWS relationship can be useful for query 3.
\item Indexing forums by id will be useful for performing efficient lookups 
for query 4.
\end{itemize}

We implement each query is implemented as a graph algorithm over the graph defined
by the indexed data.

\textbf{Query1}: A breadth first search (BFS) of the
person\_knows\_person graph based on the constraint on number of replies
x.  Each (x,y) edge in this graph stores how many replies x and y have
given to each other (these are stored while the data is loaded). Edges
with less than x replies are pruned by the BFS.

\textbf{Query2}: Add all tags to a priority queue where the order of
the tags is based on the size of the largest connected component of
the induced graph. Each tag node stores a list of persons that are
interested in the tag, and each person node stores the birthday. With
these the induced graph is created and the size of the connected
component is computed using a BFS.
              
\textbf{Query3}: Places\_Named\_Asis used to find all placeIDs with
the given name. For each of these, find all persons located at p and
add them to the induced graph. To find these persons, a person node
stores all places where the person works/studies. For each of these
places, Place\_Located\_At is used recursively to see if any place in
the resulting hierarchy is contained in the place specified by the
query. If it does, the person is added to the induced graph.
              
Then, the similarity score of all possible pairs of persons in this
graph is computed and these are added to a priority queue. To speed up
the similarity computation, tag nodes contain a hashset of the persons
interested in that tag node. Therefore the similarity between two
persons can be computed in linear time on the number of interest tags.
              
\textbf{Query4}: First a linear search is used to find the tag with
the given name. Then the induced graph of persons that are member of
forums with this tag is created. To speed up this task, each tag node
has a list of all such persons that is created when the data is first
loaded.
               
On the induced graph the centrality score of each person is computed
by using a BFS. Every time a node is expanding during the BFS, the
algorithm checks if the best possible centrality this person can
achieve is smaller than the k-th best centrality seen so far. If it
is, the BFS is stopped.
    
We are ahead of schedule; all queries are implemented and we have
submitted an implementation to Sigmod.  Our results can be seen on the
leaderboard\footnote{\url{http://www.cs.albany.edu/~sigmod14contest/leaders.html}} under team name `shparg'.

Progress on Project Milestones:
\begin{itemize}
\item Early March: Download datasets; setup Neo4j library;
  design of pipeline; implementation of queries 1 and 2 on small test dataset. \textbf{DONE}
\item March 20th: Implementation of queries 3 and 4 on small test dataset. \textbf{DONE}
\item April 1st: Submit midterm report. \textbf{DONE}
\item April 8th:  Run experiments with the larger dataset. \textbf{DONE} \\Refine the implementation to address scalability issues.
\item April 15th: Submit system to SIGMOD if we have a competitive entry. \textbf{DONE}
\item April 29th: Present results to class.
\item May 8th: Final report.
\end{itemize}

We will devote our remaining time to improving performance and
addressing scalability issues.  Our main idea is to use a disk-based
extensible hash to index the nodes and edges over the original
(unsorted) heap files\footnote{which implies we will be using
  Alternative 3 for the index data entries.} so that we may avoid
loading the entire dataset in to memory.  This index is appropriate
because we are only concerned with equality searches.

\bibliographystyle{abbrv}
\bibliography{midterm}

\end{document}
