\documentclass{article}

\usepackage{natbib}

\title{Project Proposal}

\author{
James Atwood and Luis Pineda \\ % alphabetical order
}

\begin{document}
\maketitle

We will be working on the SIGMOD 2014 programming challenge.  Briefly, in this challenge we are provided with a large (relational) social network data set and asked to implement four queries related to the graph structure induced by this data.  The motivation for choosing this project is twofold; first, there is a large (and quickly increasing) volume of graph-structured data available today, and second, developing efficient mechanisms for representing and querying graph data is a challenge research problem that is currently the subject of considerable research in the database community.

Traditionally, research in databases has focused on the relational model first proposed by Codd \cite{codd1970relational}.  This model becomes awkward and inefficient when applied to graph data, particularly for queries related to complex structure (i.e., requiring more than nearest neighbors).  For an example, please see \cite{he2008graphs} Figures 1 and 2.  More recent work has proposed other data models and query languages that more appropriately capture the rich structure evident in graph data; for instance, \cite{he2008graphs,sun2012efficient,low2010graphlab}.  However, much of this work is task oriented; for example, a system may optimize for path-related queries at the expense of subgraph isomorphism.  For the task at hand, it is unclear which existing technologies, if any, provide the best performance for the queries of this challenge.

Accordingly, we propose the following approach.  Our primary data abstraction will be an adjacency matrix over the nodes which constitute a network.  The nodes will themselves be an interface for the data provided by the challenge.  We will be investigating the particulars of the implementation of the adjacency matrix and node abstractions throughout the project.

To be more concrete, a node could be an object representing a person, with fields for attributes like gender and age.  Or, a node could simply be a pointer to a query which retrieves the node's data from disk.  The adjacency matrix could be a simple $n$ by $n$ array, where $n$ is the total number of people in the dataset.  This adjacency matrix representation scales poorly; we will probably need to employ some sparse representation or other compression to maintain this structure in memory, or develop some method for managing adjacency matrices that are only partially stored in memory.

\newpage

We set the following milestones:

\begin{itemize}
\item March 6th: Interface design (write interfaces in Java).
\item March 13th: Simple implementations of adjacency matrix and node, load small dataset.
\item March 20th: Implement an initial version of the first two queries.
\item March 27th: Implement an initial version of the second two queries. Submit midterm report.
\item April 8th: Run experiments with the larger dataset. Refine the implementation to address scalability issues.
\item April 15th: Submit system to SIGMOD if we have a competitive entry.
\item April 29th or so: Present results to class.
\item May 8th: Final report.
\end{itemize}

\bibliographystyle{named}
\bibliography{proposal}

\end{document}
